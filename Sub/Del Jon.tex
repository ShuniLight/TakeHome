\section{Reeksamen febuar 2015 opgave 3}

Lad R, S og T være binære relationer på mængden {1, 2, 3, 4}.

\subsection{Lad R = {(1, 1), (2, 1), (2, 2), (2, 4), (3, 1), (3, 3), (3, 4), (4, 1), (4, 4)}.}

Er R en partiel ordning?
\newline

For at R er en partiel ordning kræver det at den er:
    \begin{itemize}  
         \item Reflexiv
         \item Anti Symetrisk
         \item Transitativ
    \end{itemize}

Vi kan tjekke om den er reflexiv via tjekke om hvert element i vores mængde relatere til sig selv.
Vi kan derfor tjekke om 
\[\forall a \in \textbf{Z} : a = a | 1 \leq a \leq 4\]
Da det udsagn er sandt kan vi bekræfte at vores relation er reflexiv\newline
    
Det næste vi skal kan tjekke er om den er Anti Symetrisk. Det kan gøres ved at tjekke hvis element $a$ relatere til $b$ så skal element $b$ ikke relatere til element $a$

Vi kan se det via udsagnet
\[\nexists a,b \in \textbf{Z} : (a,b) \wedge (b,a) | 1 \leq a,b \leq 4\]
Da vi kan se at udsagnet er sandt ved vi at relationen er Anti Symetrisk\\

Det sidste vi skal tjekke er omkring vores relation er transitiv. Det kan gøres ved at se om hvis der er et element $a$ som relatere til element $b$ og element $b$ reletere til element $c$ så skal element $a$ også reletere til element $c$
Det kan gøres via følghende udsagn
\[\forall a,b,c \in \textbf{Z} : (a,b) \wedge (b,c) \Rightarrow (a,c) | 1 \leq a,b,c \leq 4\]
Da den også er transitiv kan vi se at den møder de tre krav vi havde for at det skulle være en partiel ordning og kan derfor bekræfte at $R$ er en partiel ordning

\subsection{Lad S = {(1, 2), (2, 3), (2, 4), (4, 2)}.}

Angiv den transitive lukning af S.\\

For at $S$ er en transitiv realation kræver det at følgende udsagn er sandt
\[\forall a,b,c \in \textbf{S} : (a,b) \wedge (b,c) \Rightarrow (a,c)\]
Vi kan dog se at dette udsagn ikke holder da vi kan se at $(1,2) \wedge (2,3) \Rightarrow (1,3)$ og $(1,2) \wedge (2,4) \Rightarrow (1,4)$ begge er falske. Vi kan derfor konkludere at den transitive lukning af $S$ er 
\[t(S) = S \cup \\\{(1,3),(1,4)\\\}\]

\subsection{Lad T = {(1, 1), (1, 3), (2, 2), (2, 4), (3, 1), (3, 3), (4, 2), (4, 4)}}
Bemærk, at T er en ækvivalens-relation.

Angiv T's ækvivalens-klasser.\\

T's ækvivalens-klasser er

\[[1] = \\\{(1,1),(1,3)\\\}\]

\[[2] = \\\{(2,2),(2,4)\\\}\]

\[[3] = \\\{(3,3),(3,1)\\\}\]

\[[4] = \\\{(4,4),(4,2)\\\}\]

Husk pakke usepackage{amssymb}
