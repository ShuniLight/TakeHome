\documentclass{article}

\begin{document}
\section{Opgave 4}

Betragt funktionen \textit{f}: $\mathbb{R} \rightarrow \mathbb{R}$ og \textit{g}: $\mathbb{R} \rightarrow \mathbb{R}$ defineret ved

$f(x)=x^2+x+1$ 

$g(x)=2x-2$ \vspace{5mm} \newline 
a) Er \textit{f} en bijektion? \vspace{5mm} \newline b) Har \textit{f} en invers funktion?\vspace{5mm} \newline c) Angiv $f+g$ \vspace{5mm} \newline d) Angiv $g \circ f$
\section{Svar}
a) Nej for selv om at f er surjektiv er den ikke injektiv. For at være bijektiv skal den være begge. \vspace{5mm} \newline b) f har ikke en invers funktion da definitionen for at finde den inverse funktion siger at den kun kan findes, hvis og kun hvis, funktionen er bijektiv.\vspace{5mm} \newline c) $f(x)+g(x) = (x^2 + x + 1) + (2x - 2) = x^2 + 2x + x -1 = x^2 + 3x - 1 \vspace{5mm} \newline$ d) $g(f(x)) = 2(x^2 + x +1) - 2 = 2x^2 + 2x$

\end{document}

\end{document}
